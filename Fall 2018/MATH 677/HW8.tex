\documentclass[12pt,letterpaper,reqno]{amsart}
\usepackage{enumerate}
\usepackage[shortlabels]{enumitem}
\usepackage{graphicx}
\usepackage{amssymb}
\usepackage[normalem]{ulem}
\usepackage{titlesec,bbm, hyperref}
\usepackage{spverbatim} 
\usepackage{esvect}
\usepackage{geometry}
\usepackage{caption}
\usepackage{subcaption}
\geometry{letterpaper, portrait, margin=1in}

\newcommand{\R}{\mathbb R}
\newcommand{\Q}{\mathbb Q}
\newcommand*{\pd}[3][]{\ensuremath{\frac{\partial^{#1} #2}{\partial #3}}}

\begin{document}

\thispagestyle{empty}
\centerline{\Large Math 677 Homework 8}
\centerline{Sumanth Ravipati}
\centerline{11/20/2018}
\vspace{.25in}

\begin{enumerate}
\item[(23)] Model from the Theory of the Spinning Tops: Consider the system

$$\alpha \dot{x} = (\beta - \gamma)yz$$
$$\beta \dot{y} = (\gamma - \alpha)zx$$
$$\gamma \dot{z} = (\alpha - \beta)xy$$

where $0 < \alpha < \beta < \gamma$

\begin{enumerate}
    \item Find all equilibrium solutions.
    \item Justify that in none of these equilibria we can apply the principle of linearized stability.
    \item Let $V_1(x,y) = \alpha(\gamma - \alpha)x^2 + \beta(\gamma - \beta)y^2$, $V_2(y,z) = \beta(\beta - \alpha)y^2 + \gamma(\gamma - \alpha)z^2$.
    \item Using (c), describe the phase portrait of the system completely.
\end{enumerate}

\item[(24)] Catalytic Hypercycle: Consider the catalytic hypercycle system given by
$$\dot{x_1} = x_1 \cdot (x_n - \Phi(x))$$
$$\dot{x_2} = x_2 \cdot (x_1 - \Phi(x))$$
$$\vdots$$
$$\dot{x_n} = x_n \cdot (x_{n-1} - \Phi(x))$$
where $\Phi(x) = x_1x_2 + x_2x_3 + \ldots + x_{n-1}x_n + x_nx_1$. Furthermore let
$$S_n = \{x \in \R^n : \sum\limits_{k=1}^n x_i = 1 \text{ and } x_k \geq 0 \text{ for all } k = 1, \ldots, n\}$$
Condsider this system for $n=2, 3$ only.
\begin{enumerate}
    \item Show that $S_n$ is invariant for (1).
    \item Determine all equilibrium points of (1) on $S_n$.
    \item For every equilibrium of (1) on $S_n$, find the tangent directions of the stable, unstable, and center manifolds at the equilibrium, provided they exist.
\end{enumerate}
\end{enumerate}
\end{document}