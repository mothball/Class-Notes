\documentclass{article}
\input xy
\xyoption{all}

\oddsidemargin=0pt
\evensidemargin=0pt
\topmargin=0in

\usepackage{geometry}
\usepackage{latexsym}
\usepackage{amsmath}
%\usepackage{showkeys}
\usepackage{amssymb}
\usepackage{amscd}
%\usepackage{psfig}
\usepackage{multicol}
\usepackage{array}
\usepackage{amsfonts}
\usepackage[dvips]{color}
\usepackage{palatino}
\usepackage{euler}
\usepackage{graphicx}
\usepackage{hyperref}
\hypersetup{
    colorlinks,
    citecolor=green,
    filecolor=black,
    linkcolor=blue,
    urlcolor=blue
}
%\usepackage{times}

\geometry{letterpaper, portrait, margin=1in}
\linespread{1.075}
\setcounter{secnumdepth}{1}
\setlength{\parindent}{0pt}\setlength{\parskip}{7pt}
%
\begin{document}
%
\centerline{\Large Abstract Algebra (MATH 321): Problem Set \#5A}
\centerline{Sumanth Ravipati}
\centerline{9/24/2018}
\vspace{.25in}

\begin{enumerate}
    \item What do you think of the subgroup tests given in this chapter? Compare and contrast.
    \begin{flushleft}
    Of the three subgroup tests introduced in this chapter, 2 were for any given subgroup and one was specific to finite subgroups. The way that they were all phrased suggested that all three could be used to determine subgroups, some with fewer requirement than the others. For example, the one-step subgroup test seemed to be the most compact, since if $ab^{-1}$ is in $H$, it implies a lot of facts about $H$ without explicitly stating them. The two-step subgroup test seems to simplify and expand upon the first in a way that seems more natural to conceptualize. The finite subgroup test has further assumptions and implications with only one property needing to be satisfied. It seems deceptively simple since only closure is needed to test for and a greater implication can be derived. Also, we can look at the brevity of the corresponding proofs to see how the second two tests leverage the first to make most of their arguments.
    \end{flushleft}
    \item Why do you think the word “cyclic” is used?
    \begin{flushleft}
        A group is called cyclic if there exists an element which serves as the generator for the entire group. This is denoted by the generator raised to various powers and if the set is finite, the various powers will start to repeat themselves. This can be directly compared to the additive group of integers modulo n, where the cyclic group has order n. In order to extend the same terminology, even a group with infinitely many generated elements, is referred to as cyclic with an infinite length.
    \end{flushleft}
    \item What do you think of the word “generator” in this context?
    \begin{flushleft}
    In the context of cyclic groups, the word generator seems apt as it captures the idea that the one element creates the remaining members. A generator communicates the fact that it can create the entire set seemingly from "scratch".
    \end{flushleft}
    \item Why is modular arithmetic important in understanding cyclic groups?
    \begin{flushleft}
    The additive group of integers modulo n is isomorphic to finite cyclic groups of order n. Any insight gained with regards to the former can be applied to the latter and due to our existing familiarity to modular arithmetic, we know more about cyclic groups than we think.
    \end{flushleft}
    \item Give at least one reason why D4 is not cyclic.
    \begin{flushleft}
    The group D4 has two distinct elements that cannot be generated by the other. There are 8 elements in the group but the greatest order of any of the elements is the identity element with order 4.
    \end{flushleft}
    \item Four explicit questions or comments on what you found interesting or confusing.
        \begin{enumerate}
            \item Both the center of a group, Z(G) and the centralizer of an element, C(a) seem to be too similar in name and concept to be able to keep straight at first glance.
            \item It would be nice to get some extra motivation behind the introduction of the Euler phi function. We are only shown that it is used to count the number of elements of each order in a finite cyclic group. The reader could be assumed to be able grasp Euler's theorem.
            \item As stated on page 76, non-examples can serve as great learning tools, sometimes even more so than representative examples. I wish this trend continues throughout the textbook.
            \item The last sentence of chapter 4 definitely leaves the reader intrigued about the possible connections from cyclic groups. The analogy to foundational concepts such as primes and chemical elements seems unexpected.
        \end{enumerate}
\end{enumerate}
\end{document} 