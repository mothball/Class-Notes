\documentclass{article}
\input xy
\xyoption{all}

\oddsidemargin=0pt
\evensidemargin=0pt
\topmargin=0in

\usepackage{geometry}
\usepackage{latexsym}
\usepackage{amsmath}
%\usepackage{showkeys}
\usepackage{amssymb}
\usepackage{amscd}
%\usepackage{psfig}
\usepackage{multicol}
\usepackage{array}
\usepackage{amsfonts}
\usepackage[dvips]{color}
\usepackage{palatino}
\usepackage{euler}
\usepackage{graphicx}
\usepackage{hyperref}
\hypersetup{
    colorlinks,
    citecolor=green,
    filecolor=black,
    linkcolor=blue,
    urlcolor=blue
}
\newcommand{\Z}{\mathbb Z}
\newcommand{\C}{\mathbb C}

\geometry{letterpaper, portrait, margin=1in}
\linespread{1.075}
\setcounter{secnumdepth}{1}
\setlength{\parindent}{0pt}\setlength{\parskip}{7pt}
%
\begin{document}
%
\centerline{\Large Abstract Algebra (MATH 321): Problem Set \#9B}
\centerline{Sumanth Ravipati}
\centerline{10/27/2018}

\begin{enumerate}
    \item[\#20.] Use Corollary 2 of Lagrange's Theorem (Theorem 7.1) to prove that the order of $U(n)$ is even when $n > 2$.
    \begin{flushleft}
    Corollary 2 of Lagrange's Theorem states that in a finite group, the order of each element of the group divides the order of the group. $U(n)$ is the set of all positive integers less than $n$ and relatively prime to $n$ under multiplication modulo $n$. For any given group $U(n)$, $n-1$ is an element since it is relatively prime to $n$. Also $(n-1)^2 = n^2 -2n +1 \equiv 1 \mod n$ and so the order of $n-1$ is 2. Therefore, the order of $U(n)$ must divide the order of $n-1$ and so $|U(n)|$ is even. $\Box$
    \end{flushleft}
    
    \item[\#27.] Let $|G| = 15$. If $G$ has only one subgroup of order 3 and only one of order 5, prove that $G$ is cyclic. Generalize to $|G| = pq$, where $p$ and $q$ are prime.
    \begin{flushleft}
    Since $15 > 5 + 3 + 1$, there must be elements of $G$ that are not in either subgroup and not the identity element. By Lagrange's theorem, the order of that element $a$ must be either 3, 5, or 15. $a$ cannot have an order of 3 or 5 since if it did, $a$ would be part of the corresponding subgroup. Therefore $|a| = 15$ and therefore $G$ is cyclic since there exists an element $a$ that can generate the entire group. $\Box$\newline
    
    This argument can be extended to a general group of order $pq$ as follows. Since $pq > p + q + 1$, there must be elements of $G$ that are not in either subgroup. By Lagrange's theorem, the order of that element $a$ must be either $p$, $q$, or $pq$. $a$ cannot have an order of $p$ or $q$ since if it did, $a$ would be part of the corresponding subgroup. Therefore $|a| = pq$ and therefore $G$ is cyclic since there exists an element $a$ that can generate the entire group. $\Box$
    \end{flushleft}
    
    \item[\#32.] Determine all finite subgroups of $\C^*$, the group of nonzero complex numbers under multiplication.
    \begin{flushleft}
    All finite subgroups of order $n$ must be generated by elements that when raised to the power of $n$ equal the identity, 1. In equation form, this is written as $x^n = 1$, the solutions of which are simply the roots of unity for a given $n$. Geometrically, we can consider subdividing the unit circle into $n$ slices. In equation form, we write this as $x = e^{2\pi i/n}$ since $(e^{2\pi i/n})^n = e^{2\pi i} = 1$. In component form, we can write $e^{2\pi i/n}$ as $\cos(2\pi/n) + i \sin(2\pi/n)$.
    \end{flushleft}
    
    \item[\#42.] Let $G$ be a group of order $n$ and $k$ be any integer relatively prime to $n$. Show that the mapping from $G$ to $G$ given by $g \rightarrow g^k$ is one-to-one. If $G$ is Abelian, show that the mapping given by $g \rightarrow g^k$ is an automorphism of $G$.
    \begin{flushleft}
    To prove that the mapping is one-to-one, we must show that if $a^k = b^k$ for any elements in $G$, then $a = b$. Since $n$ and $k$ are relatively prime, there exist integers $x$ and $y$ such that $xn + yk = 1$. Therefore $a = a^1 = a^{xn + yk} = a^{xn}a^{yk}$. Since the order of $G$ is $n$, $a^n = e$ and so $a^{xn} = (a^n)^x = e^x = e$. Therefore, $a^{xn}a^{yk} = a^{yk} = (a^k)^y = (b^k)^y$ since we have assumed that $a^k = b^k$. Since $b^n = e$, $(b^k)^y = e^{x}(b^k)^y = b^{xn}b^{yk} = b^{xn + yk} = b^1 = b$. Therefore $a^k = b^k$ implies that $a = b$, as desired. $\Box$\newline
    
    To show that the mapping $g \rightarrow g^k$ for a given $k$ is onto, for any given element $b$ in $G$, there must exist an element $a$ in $G$ such that $a^k = b$. Since $k$ is relatively prime to $n$, using the same argument as above, $b = b^1 = b^{xn + yk} =$ $b^{xn}b^{yk} = (b^n)^x(b^y)^k =$ $e^x(b^y)^k =$ $(b^y)^k$. Since $G$ is closed under the operation, we can set $a = b^y$ and so we have found an element $a$ in $G$ such that $a^k = b$, as desired.\newline
    
    We will prove that $(ab)^k = a^kb^k$ via induction so that the mapping $g \rightarrow g^k$ preserves the group operation. To show the base case, observe that $(ab)^1 = ab = a^1b^1$, as desired. If we assume that $(ab)^{k-1} = a^{k-1}b^{k-1}$, we shall show that $(ab)^k = a^kb^k$. If $G$ is Abelian, $(ab)^k = (ab)^{k-1}(ab) = $ $a^{k-1}b^{k-1}(ab) = a^{k-1}b^{k-1}(ba) = a^{k-1}b^ka = a^{k-1}ab^k = a^kb^k$, as desired. Therefore $(ab)^k = a^kb^k$ via the principle of mathematical induction.\newline
    
    Therefore if $G$ is Abelian, the mapping $g \rightarrow g^k$ is an automorphism of $G$ since it is one-to-onto, onto and preserves the group operation. $\Box$
    \end{flushleft}
    
    \item[\#8.] Is $\Z_3 \bigoplus \Z_9$ isomorphic to $\Z_{27}$? Why?
    \begin{flushleft}
    If we let $m = n_1n_2 \ldots n_k$, then $\Z_m$ is isomorphic to $\Z_{n_1} \bigoplus \Z_{n_2} \bigoplus \ldots \bigoplus \Z_{n_k}$ if and only if $n_i$ and $n_j$ are relatively prime when $i \not= j$. Since $27 = 3 \cdot 9$ but 3 is not relatively prime to 9, $\Z_3 \bigoplus \Z_9$ is not isomorphic to $\Z_{27}$.
    \end{flushleft}
\end{enumerate}
\end{document} 