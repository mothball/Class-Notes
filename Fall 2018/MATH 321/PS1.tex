\documentclass{article}
\input xy
\xyoption{all}

\oddsidemargin=0pt
\evensidemargin=0pt
\topmargin=0pt
\usepackage{geometry}
\geometry{letterpaper, portrait, margin=1in}

\usepackage{latexsym}
\usepackage{amsmath}
\usepackage{amssymb}
\usepackage{amscd}
\usepackage{multicol}
\usepackage{array}
\usepackage{amsfonts}
\usepackage[dvips]{color}
\usepackage{palatino}
\usepackage{euler}
\usepackage{graphicx}


\newcommand{\C}{\mathbb{C}}
\newcommand{\Q}{\mathbb{Q}}
\newcommand{\R}{\mathbb{R}}



\linespread{1.075}
\setcounter{secnumdepth}{1}
\setlength{\parindent}{0pt}\setlength{\parskip}{7pt}
%
\begin{document}
%
\title{Abstract Algebra (MATH 321): Problem Set \#1}
\author{Sumanth Ravipati}
\date{August, 27, 2018}
\maketitle

\begin{enumerate}
\item What do you think the word ``relation'' means in the phrase ``equivalence relation"?

\begin{flushleft}
A relation conveys a relationship among several items. With respect to the phrase ``equivalence relation", it means a relation with all 3 of the following properties: reflexivity, symmetry, and transitivity.
\end{flushleft}

\item Give an example of a relation that is not an equivalence relation. Which part of the definition fails?

\begin{flushleft}
An equivalence relation must have the following properties:
\begin{enumerate}
    \item reflexive: $ \forall a \in E, a \sim a $
    \item symmetric: $ \forall a,b \in E, a \sim b \Rightarrow b \sim a $
    \item transitive: $ \forall a,b,c \in E, (a \sim b) \wedge (b \sim c) \Rightarrow a \sim c $
\end{enumerate}
The relation $\leq$ with respect to the set of integers is both reflexive and transitive but not necessarily symmetric, which would mean that it is not an equivalent relation:
\begin{enumerate}
    \item reflexive: $ \forall a \in \mathbb{Z}, a \leq a $
    \item not symmetric: $ \forall a,b \in \mathbb{Z}, a \leq b \not\Rightarrow b \leq a $
    \begin{flushleft}
        counter-example: $ 1 \leq 2 $ but $ 2 \not\leq 1 $
    \end{flushleft}
    \item transitive: $ \forall a,b,c \in \mathbb{Z}, (a \leq b) \wedge (b \leq c) \Rightarrow a \leq c $
\end{enumerate}
The symmetry property of equivalence relation fails in this example.
\end{flushleft}

\item What is the connection between partitions and equivalence relations?
\begin{flushleft}
There is a well known theorem that states that if we have an equivalence relation R on A, the equivalence classes of that relation form a partition of A. The same rules for a relation that satisfy the equivalence relation, satisfy the conditions for a partition. Given a partition, we can define an equivalence relation and given an equivalence relation, we can define a partition as follows:
\begin{enumerate}
    \item Given a partition of a set $S$ in to subsets $A_1, A_2, \ldots$, the equivalence relation is defined as: $x \sim y \Leftrightarrow$ ``x and y belong to the same set $A_i$ from the partition" This satisfies the conditions for an equivalence relation as follows:
    \begin{enumerate}
        \item reflexive: $a \in A_i \Leftrightarrow a \in A_i$
        \item symmetric: $a_1,a_2 \in A_i \Leftrightarrow a_2,a_1 \in A_i$
        \item transitive: $(a_1,a_2 \in A_i) \wedge (a_2,a_3 \in A_i) \Leftrightarrow a_1,a_3 \in A_i$
    \end{enumerate}
    \item Given an equivalence relation on a set $S$, it satisfies the conditions for a partition as follows:
    \begin{enumerate}
        \item pairwise disjoint: for any $x,y \in S$, the associated equivalence classes $E_x, E_y$ are either equal or disjoint.
        \item union is all of S: for any $x \in S$, there exists an equivalence class that contains $x \, (E_x)$.
    \end{enumerate}
\end{enumerate}
\end{flushleft}

\item  Make sense of the following statement: ``A function from $A$ to $B$ is just a $B$-labeled partition of $A$." What do you think of this statement?

\begin{flushleft}
A function maps elements of A onto elements of B. In particular, a function has a particular value in B for each element in A. Each element in A can only have one corresponding value in B but unless it is one-to-one, several values in A can map to the same value in B. Each subset of A that maps to a particular value in B, it can be thought of as a partition that is labeled with its co-domain, the particular value in B. Such subsets satisfy the conditions for partitions and equivalence relations as follows:
\begin{enumerate}
    \item pairwise disjoint: an element in A maps to a particular element in B so it cannot belong to 2 B-labeled partitions simultaneously.
    \item union is all of A: the union of all such subset will equal to all of A since every element in A maps to an element in B.
    \item reflexive: $a \in A \to b \in B \Leftrightarrow a \in A \to b \in B$, it is in the same subset of $A$ as itself. Namely, $\{a\} \subset \{a\}$.
    \item symmetric: $a_1,a_2 \in A \to b \in B \Leftrightarrow a_2,a_1 \in A \to b \in B$
    \item transitive: $(a_1,a_2 \in A \to b \in B) \wedge (a_2,a_3 \in A \to b \in B) \Leftrightarrow a_1,a_3 \in A \to b \in B$
\end{enumerate}
\end{flushleft}

\item What is the difference in your opinion between the first and second principles of mathematical induction? Why do you think anyone bothered with the second principle?

\begin{flushleft}
There are circumstances where assuming a statement is true for a given n, tells us nothing directly about if it is true for n+1. In these situations, it can be helpful to assume that the statement is true for a range of integers $a \leq x \leq n$. The second principle of mathematical induction also doesn't require a base and one can assume more in the inductive step.
\end{flushleft}

\item Why aren't the set of integers ``well-ordered"? And why do you think this matters for induction?

\begin{flushleft}
A set is well-ordered if there exists a smallest element in the set. In other words, every other element must be greater than the smallest element. Let's do a simple proof by contradiction: \newline
Suppose $\exists \, x \in \mathbb{Z}$ s.t. $x < y \, \forall \, y \in \mathbb{Z} \setminus \{x\}$. By the closure property of the integers, $x + (-1) \in \mathbb{Z}$. However, $x - 1 \leq x$, which contradicts the statement that $x$ is the smallest element in $\mathbb{Z}$. Therefore there is no smallest element of $\mathbb{Z}$ and so the set of integers are not ``well-ordered".
\end{flushleft}

\item Give an example that illustrates the theorem that the GCD of two numbers is a linear combination of these two numbers.Try to think of a non-trivial example. A non-trivial example is an example that illustrates the principle in a way that you wouldn't have already known without the principle. For example, if you say $gcd(10,10) = 0\cdot 10+1\cdot 10$ then you have given what I would consider a trivial example.

\begin{flushleft}
$$ gcd(7,13) = 1 = a \cdot 7 + b \cdot 13 $$
There exists at least one solution with one explicitly listed below:
$$ gcd(7,13) = 1 = 2 \cdot 7 + (-1) \cdot 13 $$
As a simple proof by counterexample, we show how the solutions to the linear combination equation are not unique as both (2, -1) and (-11, 6) are both solutions.
$$ 2 \cdot 7 + (-1) \cdot 13 = (-11) \cdot 7 + 6 \cdot 13 $$
\end{flushleft}

\item Feedback: Come up with a couple of explicit questions or comments on what you found interesting or confusing. Write them down.

\begin{flushleft}
\begin{enumerate}
    \item It would be interesting to see more examples of proofs that require the second principle of mathematical induction (where the first principle wouldn't suffice).
    \item Are mappings between sets that are not functions, not as interesting to study? What about mappings that are not functions but onto or one-to-one?
\end{enumerate}
\end{flushleft}

\end{enumerate}
\end{document} 