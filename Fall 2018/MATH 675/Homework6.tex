\documentclass[12pt,letterpaper,reqno]{amsart}
\usepackage{enumerate}
\usepackage[shortlabels]{enumitem}
\usepackage{graphicx}
\usepackage{amssymb}
\usepackage[normalem]{ulem}
\usepackage{titlesec,bbm, hyperref}
\usepackage{spverbatim} 

\newcommand{\R}{\mathbb R}
\newcommand{\Q}{\mathbb Q}

\begin{document}

\thispagestyle{empty}
\centerline{\Large Math 675 Homework 6}
\centerline{Due 10/3/2018}
\vspace{.25in}

\begin{enumerate}[1.]
\item Give an example of a contraction that does not have a fixed point.
\item The contraction theorem requires a mapping that is $L$-Lipschitz for $L<1$. Prove that the result is sharp. That is, allowing $L=1$ in the theorem would make it false.
\item Prove that a closed subset of a complete space is complete.
\item Use the Picard Theorem to find a few approximate solutions to the differential equation $\frac{dy}{dx}=y$, with $y_0=1$, $x_0=0$, and initial guess of $\phi_0(x)=0$. Then write down the solution as a series.
\item Prove that a closed subset of a compact space is compact. (Hint: its complement is an open set.)
\item Let $A, B\subset X$ be two compact subsets of a metric space $X$. Prove that $A\cup B$ and $A\cap B$ are both compact.
\end{enumerate}
\end{document}