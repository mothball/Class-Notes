\documentclass[12pt,letterpaper,reqno]{amsart}
\usepackage{enumerate}
\usepackage[shortlabels]{enumitem}
\usepackage{graphicx}
\usepackage{amssymb}
\usepackage[normalem]{ulem}
\usepackage{titlesec,bbm, hyperref}
\usepackage{spverbatim} 

\newcommand{\R}{\mathbb R}
\newcommand{\Q}{\mathbb Q}

\begin{document}

\thispagestyle{empty}
\centerline{\Large Math 675 Homework 5}
\centerline{Due 9/26/2018}
\vspace{.25in}
Warning: these problems are more tedious than hard, and mostly come down to unwrapping all of the definitions.

\begin{enumerate}[1.]
\item Suppose two metric spaces $X$ and $Y$ are bi-Lipschitz equivalent. Prove that that $X$ is complete if and only if $Y$ is complete.
\item Give an example of homeomorphic metric spaces $X$ and $Y$ such that $X$ is complete but $Y$ isn't. (Hint: $\arctan$.)
\item Following Example 5 in Section 7.1, prove that $\ell_\infty$ is complete.
\item Let $X$ be a metric space, and $Y$ the completion of $X$ defined in class. Let $\{y_i\}$ be a sequence of points of $Y$, and for each $i$ let $\{x_i^j\}_{j=1}^\infty$ be a Cauchy sequence representing. 
\begin{enumerate}[(i)]
\item  Prove that the diagonal sequence $z_i=x_i^i$ is Cauchy.
\item Let $y$ be the limit point of $z_i$. Prove that the sequence $y_i$ converges to $y$.
\end{enumerate}
\end{enumerate}





\end{document}












































































