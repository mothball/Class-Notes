\documentclass[12pt,letterpaper,reqno]{amsart}
\usepackage{enumerate}
\usepackage[shortlabels]{enumitem}
\usepackage{graphicx}
\usepackage{amssymb}
\usepackage[normalem]{ulem}
\usepackage{titlesec,bbm, hyperref}
\usepackage{spverbatim} 
\usepackage{esvect}
\usepackage{geometry}
\usepackage{caption}
\usepackage{subcaption}
\geometry{letterpaper, portrait, margin=0.4in}

\newcommand{\R}{\mathbb R}
\newcommand{\Q}{\mathbb Q}
\newcommand{\N}{\mathbb N}

\begin{document}

\thispagestyle{empty}
\centerline{\Large Math 675 Homework 4}
\centerline{Sumanth Ravipati}
\centerline{9/18/2018}
\vspace{.15in}
\begin{enumerate}[1.]
\item Prove that the following are equivalent for a function $f: X\rightarrow Y$ between two metric spaces:
\begin{enumerate}[(i)]
\item $f$ is continuous,
\item $f^{-1}(A)$ is an open set for every open set $A\subset Y$,
\item $f^{-1}(B)$ is a closed set for every closed set $B\subset Y$.
\end{enumerate}
\begin{flushleft}
$(i) \Rightarrow (ii)$: We are given that $A \subset Y$ is an open set and that $f$ is continuous. For a given $a \in f^{-1}(A)$, we know that $f(a) \in A$. Given that $A$ is open, $\exists \epsilon > 0$ such that $B_\epsilon(f(a)) \subset A$. Since $f$ is continuous, $\exists \delta > 0$ such that $d(x,a) < \delta \Rightarrow d(f(x),f(a)) < \epsilon$. Analogously, $x \in B_\delta(a) \Rightarrow f(x) \in B_\epsilon(f(a))$. Therefore, $f(x) \in A \Rightarrow x \in f^{-1}(A)$. Since $x \in B_\delta(a) \Rightarrow x \in f^{-1}(A)$, $B_\delta(a) \subset f^{-1}(A)$, and so $f^{-1}(A)$ is an open set. \newline
$(ii) \Rightarrow (i)$: We are given that $f^{-1}(A)$ is an open set for every open set $A\subset Y$. As before, for a given $a \in f^{-1}(A)$, we know that $f(a) \in A$. Given that $A$ is open, $\exists \epsilon > 0$ such that $B_\epsilon(f(a))$ is open. Since $f(a) \in B_\epsilon(f(a))$, $a \in f^{-1}(B_\epsilon(f(a)))$. Due to the assumption from $(ii)$, $B_\epsilon(f(a))$ is open $\Rightarrow f^{-1}(B_\epsilon(f(a)))$ is open. Given that $ f^{-1}(B_\epsilon(f(a)))$ is open, $\exists \delta > 0$ such that $B_\delta(a) \subset f^{-1}(B_\epsilon(f(a)))$. Equivalently, $x \in B_\delta(a) \Rightarrow x \in f^{-1}(B_\epsilon(f(a))) \Rightarrow f(x) \in B_\epsilon(f(a))$. In other words, $d(x,a) < \delta \Rightarrow d(f(x),f(a)) < \epsilon$. Since we can find $\delta > 0$ for any $\epsilon > 0$, this proves that $f$ is continuous. \newline
$(ii) \Rightarrow (iii)$: A subset $B$ of a metric space $Y$ is called closed if its complement is open in $Y$. For a given closed set $B \subset Y$, we are to show that $f^{-1}(B)$ is closed. If $B$ is closed, it's complement $Y \setminus B$ must be open. If $Y \setminus B$ is open, then by $(ii)$, $f^{-1}(Y \setminus B)$ is also open. \newline
Lemma 1: $f^{-1}(Y \setminus B) = X \setminus f^{-1}(B)$. \newline Proof: $a \in f^{-1}(Y \setminus B) \Leftrightarrow f(a) \in Y \setminus B \Leftrightarrow f(a) \not\in B \Leftrightarrow a \not\in f^{-1}(B) \Leftrightarrow a \in X \setminus f^{-1}(B)$. Therefore, $[f^{-1}(Y \setminus B) \subset X \setminus f^{-1}(B)$ and $X \setminus f^{-1}(B) \subset f^{-1}(Y \setminus B)] \Leftrightarrow f^{-1}(Y \setminus B) = X \setminus f^{-1}(B)$, as desired. \newline
Since $f^{-1}(Y \setminus B)$ is open, $X \setminus f^{-1}(B)$ is also open. The complement of this set, $f^{-1}(B)$, must therefore be closed. \newline
$(iii) \Rightarrow (ii)$: We are given that $f^{-1}(B)$ is a closed set for every closed set $B\subset Y$. If $A$ is an open set, its complement $Y\setminus A$ is closed and so $f^{-1}(Y\setminus A)$ is closed. As seen above, inverse images commute with complements and so, $X \setminus (f^{-1}(Y\setminus A)) = f^{-1}(Y\setminus A)$. Therefore, the complement of $f^{-1}(Y\setminus A)$ is closed, and so $f^{-1}(A)$ is open. Therefore we have shown that if $A$ is an open set, $f^{-1}(A)$ is open.\newline
We have proven equivalence by proving that: $(i) \Leftrightarrow (ii) \Leftrightarrow (iii)$. $\Box$
\end{flushleft}
\item Prove that:
\begin{enumerate}[(a)]
\item The set of invertible 2-by-2 matrices is open. (Interpret the space of all 2-by-2 matrices as $\R^4$ with the usual topology.)
\begin{flushleft}
Let $A$ denotes the set of $M_{2\times2}$ that are also invertible.
Let $det: A \rightarrow \R\setminus\{0\}$ be the determinant function for invertible 2-by-2 matrices.
\[
|A| = 
\begin{vmatrix}
    a_{11} & a_{12} \\
    a_{21} & a_{22} \\
\end{vmatrix}
= a_{11}\cdot a_{22} - a_{12}\cdot a_{21}
\]
Since $det$ is a quadratic polynomial, it is a continuous function. Using 1(ii), we know that $f^{-1}(A)$ is an open set for every open set $A\subset Y$ if $f$ is a continuous function. $\R\setminus\{0\}$ is an open set, as it is a finite union of open sets, $(-\infty, 0) \cup (0, \infty)$. Since $\R\setminus\{0\}$ is an open set, $det^{-1}(\R\setminus\{0\}) = A$ must also be an open set. $\Box$
\end{flushleft}
\item The set of determinant-one 2-by-2 matrices is closed.
\begin{flushleft}
Let $B$ denote the set of $M_{2\times2}$ that have determinant $= 1$.
Let $det: B \rightarrow \{1\}$ be the determinant function for determinant-one 2-by-2 matrices. As before, we know that the determinant function is a continuous quadratic and so we can utilize $(i) \Rightarrow (iii)$ from problem 1 to show that $B$ is a closed set. The image of the $det$ function is the closed set $\{1\}$ and so the pre-image $det^{-1}(\{1\}) = B$, must also be a closed set. $\Box$
\end{flushleft}
\end{enumerate}
\newpage
\item Let $A$ be a linearly independent set in a metric vector space $V$.
\begin{enumerate}[(a)]
\item Assuming that $V$ is finite-dimensional, prove that the set $\R A=\{\lambda a : \lambda \in \R, a\in A\}$ is closed.
\begin{flushleft}
Let A be the linearly independent set $\{a_1, \ldots, a_n \}$, where $n < \infty$. We shall prove that the set $\R A=\{\lambda a : \lambda \in \R, a\in A\}$ is closed by showing that it is not open, regardless of the neighborhood of points considered. Without loss of generality, for any given $\lambda a_1 \in \R A$, let us take a ball $B_\epsilon(\lambda a_1)$. By definition, $\lambda a_1 + \epsilon a_2$ is within $B_\epsilon(\lambda a_1)$ but is not within $\R A$ as it does not contain any linear combination of independent vectors. Alternatively, we can also say that $\R A$ contains all of its limit points as every neighborhood contains infinitely many points in $\R A$. For a given $B_\epsilon(\lambda a_1)$, there exists infinitely many $\gamma \in \R$ such that $\gamma \lambda a_1 < \epsilon \lambda a_1$. Since $\R A$ contains all of its limit points, it is closed. $\Box$
\end{flushleft}
\item Give an infinite-dimensional example for which $\R A$ is not closed.
\begin{flushleft}
Let the metric vector space $V$ be the space of infinite-dimensional sequences of the form $(a_1, \ldots, a_n, \ldots)$ with the norm of $\|(a_1, \ldots, a_n, \ldots)\| = \sum_{k=1}^\infty a_k$. Let $A$ be a linearly independent subset of the form: $A_i = (a_1, \ldots, a_i, 0, \ldots)$, with a finite number of non-zero entries. These elements are still infinite-dimensional as there are an infinite number of trailing $0$s. The elements of $A$ as constructed are linearly independent as no single element can be represented as a linear combination of the others. Now, let us consider the sequences of the form $B_k = (1, \frac{1}{2}, \frac{1}{3}, \ldots, \frac{1}{k}, \ldots)$. Each individual of the sequences is an element of $A$ and in general, $\in \R A=\{\lambda a : \lambda \in \R, a\in A\}$. However, the limit of the sequence $B_k$ is not in the set $\R A$ as it has infinitely many non-zero elements. Since every neighborhood of the limit sequence $B_k$ has infinitely many points in $\R A$, it is a limit point of the set $\R A$. Therefore $\R A$ is not closed, as it does not contain all of its limit points.
\end{flushleft}
\end{enumerate}
\item Provide examples of:
\begin{enumerate}[(a)]
\item An infinite union of closed sets that is not closed.
\begin{flushleft}
$$\bigcup_{n \in \N} A_n = \bigcup_{n \in \N} [\frac{1}{n},\infty) = (0,\infty)$$
Each $A_n = [\frac{1}{n},\infty)$ is a closed set since unbounded intervals are closed if they contain a closed finite endpoint.
\end{flushleft}
\item An infinite intersection of open sets that is not open.
\begin{flushleft}
$$\bigcap_{n \in \N} A_n = \bigcap_{n \in \N} (-\frac{1}{n},\frac{1}{n}) = \{0\}$$
Each $A_n = (-\frac{1}{n},\frac{1}{n})$ is an open set since both endpoints are open and finite. The infinite intersection $\{0\}$ is not open, since single point sets are closed in $\R$.
\end{flushleft}
\end{enumerate}
\end{enumerate}
\end{document}