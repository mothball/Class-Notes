\documentclass[12pt,letterpaper,reqno]{amsart}
\usepackage{enumerate}
\usepackage[shortlabels]{enumitem}
\usepackage{graphicx}
\usepackage{amssymb}
\usepackage[normalem]{ulem}
\usepackage{titlesec,bbm, hyperref}
\usepackage{spverbatim} 
\usepackage{tikz}
\usepackage{geometry}
\geometry{letterpaper, portrait, margin=0.4in}

\newcommand{\R}{\mathbb R}
\newcommand{\Q}{\mathbb Q}
\newcommand{\N}{\mathbb N}
\newcommand{\norm}[1]{\left\vert #1 \right \vert}	
\newcommand{\Norm}[1]{\left\Vert #1 \right \Vert}


\begin{document}

\thispagestyle{empty}
\centerline{\Large Math 675 Homework 8}
\centerline{Sumanth Ravipati - 10/30/2018}
\vspace{.25in}

\begin{enumerate}[1.]
\item Write $(1,2,3)$ as a linear combination of the vectors $(1,1,1)$, $(3,5,6)$ and $(7,8,9)$ using orthonormalization. Provide rounded numerical calculations, rather than exact ones.
\begin{flushleft}
Let us define $v_1 = (1,1,1), v_2 = (3, 5, 6)$, and $v_3 = (7, 8, 9)$. We can determine $u_1 = \frac{v_1}{\Norm{v_1}} = $ $\frac{(1,1,1)}{\sqrt{1^2+1^2+1^2}} = \frac{1}{\sqrt{3}}(1,1,1)$. We can calculate $u_2$ as follows: $u_2^\prime = v_2 - \frac{\langle u_1, v_2 \rangle}{\langle u_1, u_1 \rangle}u_1 = $ $(3, 5, 6) - \frac{1}{3}\frac{(3+5+6)}{1}(1,1,1) =$ $(\frac{9}{3}, \frac{15}{3}, \frac{18}{3}) - (\frac{14}{3}, \frac{14}{3}, \frac{14}{3}) =$ $ \frac{1}{3}(-5,1,4)$. We normalize this vector: $u_2 = \frac{u_2^\prime}{\Norm{u_2^\prime}} = \frac{(-5, 1, 4)}{\sqrt{5^2+1^2+4^2}} = \frac{1}{\sqrt{42}}(-5,1,4)$. Finally, $u_3^\prime = v_3 - \frac{\langle u_1, v_3 \rangle}{\langle u_1, u_1 \rangle}u_1 - \frac{\langle u_2, v_3 \rangle}{\langle u_2, u_2 \rangle}u_2 =$ $(7,8,9) - \frac{(7+8+9)}{3}(1,1,1) - \frac{(-35+8+36)}{42}(-5,1,4) = (7-8+\frac{15}{14}, 8-8-\frac{3}{14}, 9-8-\frac{12}{14}) =$ $\frac{1}{14}(1, -3, 2)$. Normalizing, $u_3 = \frac{u_3^\prime}{\Norm{u_3^\prime}} = \frac{\frac{1}{14}(1, -3, 2)}{\frac{1}{14}\sqrt{1^2+3^2+2^2}} = \frac{1}{\sqrt{14}}(1, -3, 2)$. Therefore the orthonormalized basis is $\{\frac{1}{\sqrt{3}}(1,1,1), \frac{1}{\sqrt{42}}(-5,1,4), \frac{1}{\sqrt{14}}(1, -3, 2)\}$. We will next find constants $a_1, a_2, a_3$ such that $(1,2,3) = a_1u_1 + a_2u_2 + a_3u_3$. We can write this equation in matrix form as follows:
$$\left[ \begin{array}{c}
    {1}\\{2}\\{3}
\end{array}\right] =
    \left[ \begin{array}{ccc}
    {\frac{1}{\sqrt{3}}} & {\frac{-5}{\sqrt{42}}} & {\frac{1}{\sqrt{14}}} \\
    {\frac{1}{\sqrt{3}}} & {\frac{1}{\sqrt{42}}} & {\frac{-3}{\sqrt{14}}} \\
    {\frac{1}{\sqrt{3}}} & {\frac{4}{\sqrt{42}}} & {\frac{2}{\sqrt{14}}}
\end{array} \right]
    \left[ \begin{array}{c}
    {a_1}\\{a_2}\\{a_3}
\end{array}\right]$$
To solve for the constants, we can rewrite the previous equation in terms of the inverse of the matrix:
$$\left[ \begin{array}{c}
    {a_1}\\{a_2}\\{a_3}
\end{array}\right] =
    \left[ \begin{array}{ccc}
    {\frac{1}{\sqrt{3}}} & {\frac{-5}{\sqrt{42}}} & {\frac{1}{\sqrt{14}}} \\
    {\frac{1}{\sqrt{3}}} & {\frac{1}{\sqrt{42}}} & {\frac{-3}{\sqrt{14}}} \\
    {\frac{1}{\sqrt{3}}} & {\frac{4}{\sqrt{42}}} & {\frac{2}{\sqrt{14}}}
\end{array} \right]^{-1}
    \left[ \begin{array}{c}
    {1}\\{2}\\{3}
\end{array}\right] =
    \left[ \begin{array}{ccc}
    {\frac{1}{\sqrt{3}}} & {\frac{1}{\sqrt{3}}} & {\frac{1}{\sqrt{3}}} \\
    {\frac{-5}{\sqrt{42}}} & {\frac{1}{\sqrt{42}}} & {\frac{4}{\sqrt{42}}} \\
    {\frac{1}{\sqrt{14}}} & {\frac{-3}{\sqrt{14}}} & {\frac{2}{\sqrt{14}}}
\end{array} \right]
    \left[ \begin{array}{c}
    {1}\\{2}\\{3}
\end{array}\right]$$
% $$\left[ \begin{array}{c}
%    {a_1}\\{a_2}\\{a_3}
%\end{array}\right] =
%\left[ \begin{array}{c}
%    {2\sqrt{3}}\\{\frac{9}{\sqrt{42}}}\\{\frac{1}{\sqrt{14}}}
%\end{array}\right] \approx
%\left[ \begin{array}{c}
%    {3.461}\\{1.389}\\{0.267}
%\end{array}\right]$$
$$[a_1, a_2, a_3] = [ 2\sqrt{3}, \frac{9}{\sqrt{42}}, \frac{1}{\sqrt{14}}] \approx [3.461, 1.389, 0.267]$$
Finally, in rounded approximate form, we have: $$(1,2,3) \approx 3.461(0.577, 0.577, 0.577) + 1.389(-0.772, 0.154, 0.617) + 0.267(0.267, -0.802, 0.535)$$
\end{flushleft}

\item Let $\Norm{\cdot}$ be the norm in $\R^2$ for which the unit circle is a regular hexagon with side length 1. Prove that $\Norm{\cdot}$ is not induced by an inner product.
\begin{flushleft}
Let $\vec{v}$ and $\vec{w}$ come from the origin as shown below. The vector $\vec{v}+\vec{w}$ is then given by the vector from the origin to the vertex in between the other 2. Finally notice that the vector that originates from the head of $\vec{w}$ and terminates at the head of $\vec{v}$ equals the difference $\vec{v} - \vec{w}$. Since $\vec{v}$, $\vec{w}$, and $\vec{v}+\vec{w}$ all lie on the unit circle(hexagon), their norms are equal to 1. The vector $\vec{v} - \vec{w}$ has a norm of 2 since the vertices labeled $v$ and $w$ are 2 units apart. We can then show that the parallelogram law is not satisfied by this norm as $2(\|v\|^2 + \|w\|^2) = 2(1+1) = 4$ while $ \| v + w \|^2 + \| v - w \|^2 = 1^2 + 2^2 = 5$. Therefore,  $2(\|v\|^2 + \|w\|^2) \not= \|v + w\|^2 + \|v - w\|^2$ and so this is norm is not induced by an inner product.
\end{flushleft}
\begin{center}
\begin{tikzpicture}
   \newdimen\R
   \R=2.7cm
   \draw (0:\R) \foreach \x in {60,120,...,360} {  -- (\x:\R) };
   \foreach \x/\l/\p in
     { 60/{}/above,
      120/{}/above,
      180/{}/left,
      240/{w = (0,1)}/below,
      300/{v+w = (1,1)}/below,
      360/{v = (1,0)}/right
     }
     \node[inner sep=1pt,circle,draw,fill,label={\p:\l}] at (\x:\R) {};
\end{tikzpicture}
\end{center}
\newpage
\item Let $f_i(x)=x^i$, $i\in \N_{\geq0}$, be the basis of monomials in $C_2[a,b]$.
\begin{enumerate}
\item Is it true that every continuous function on $C_2[a,b]$ is of the form $\sum_{i=0}^\infty a_if_i$? (Hint: take a derivative.)
\begin{flushleft}
By the Stone-Weierstrass theorem, we know that every continuous function defined on a closed interval $[a, b]$ can be uniformly approximated as closely as desired by a polynomial function. Since polynomials are simply linear combinations of monomials, we know that $f_i(x)=x^i$ spans the set of polynomials and by extension, the set of continuous functions over the interval $[a,b]$. The span of the basis of monomials can be represented by the sum $\sum_{i=0}^\infty a_if_i$. Since $C_2[a,b]$ denotes the set of twice differentiable functions, we can take the derivative of any function $h(x) \in C_2[a,b]$ and still know that $h^\prime(x)$ is a continuous function in $[a,b]$. 
\end{flushleft}
\vspace{1.5in}
\item Let $g_i$ be the associated orthonormal basis.  Is it true that every continuous function on $C_2[a,b]$ is of the form $\sum_{i=0}^\infty a_ig_i$? 
\vspace{2.5in}
\item Why don't the two results contradict each other?
\begin{flushleft}
A given vector space can be spanned by two different sets of bases.
\end{flushleft}
\end{enumerate}
\end{enumerate}
\end{document}