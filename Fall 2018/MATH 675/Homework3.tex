\documentclass[12pt,letterpaper,reqno]{amsart}
\usepackage{enumerate}
\usepackage[shortlabels]{enumitem}
\usepackage{graphicx}
\usepackage{amssymb}
\usepackage[normalem]{ulem}
\usepackage{titlesec,bbm, hyperref}
\usepackage{spverbatim} 

\newcommand{\R}{\mathbb R}
\newcommand{\Q}{\mathbb Q}

\begin{document}

\thispagestyle{empty}
\centerline{\Large Math 675 Homework 3}
\centerline{Due 9/12/2018}
\vspace{.25in}

Responses must be typed, with a two-page maximum and LaTeX highly recommended. Grade will be based on style (2 points) and all of the problems.

Let $A\subset X$ be a subset of a metric space $(X,d)$. 
\begin{enumerate}[1.]
\item Prove that $x\in X$ is a limit point of $A$ if and only if for every $\epsilon>0$ the ball $B_\epsilon(x)$ contains a point of $A$ not equal to $x$.
\item Prove that the closure of $A$ equals the union of the set of isolated points $A$ with the set of limit points of $A$.
\item Let $a<b\in \R$. Prove that the closed interval $[a,b]$ is also closed in the topological sense of the word.
\item What is the closure of $\Q\subset \R$? Prove that you are correct.
\item Look up Lipschitz functions, and notice that definition of bi-Lipschitz is a generalization of the Lipschitz condition. Another common condition in real analysis is the H\"older condition. This leads to a notion of equivalence that is intermediate between bi-Lipschitz equivalence and homeomorphic equivalence.
\begin{enumerate}[(a)]
\item Define bi-H\"older equivalence between metric spaces, analogously to the definition of bi-Lipschitz equivalence.
\item Suppose $X$ and $Y$ are bi-Lipschitz equivalent. Prove that they are also bi-H\"older equivalent.
\item Suppose $X$ and $Y$ are bi-H\"older equivalent. Prove that they are also homeomorphic.
\end{enumerate}
\end{enumerate}





\end{document}












































































