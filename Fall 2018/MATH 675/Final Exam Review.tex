\documentclass[12pt,letterpaper,reqno]{amsart}
\usepackage{enumerate}
\usepackage[shortlabels]{enumitem}
\usepackage{graphicx}
\usepackage{amssymb}
\usepackage[normalem]{ulem}
\usepackage{titlesec,bbm, hyperref}
\usepackage{spverbatim} 
\usepackage{geometry}
\geometry{letterpaper, portrait, margin=0.25in}

\newcommand{\C}{\mathbb C}
\newcommand{\R}{\mathbb R}
\newcommand{\Q}{\mathbb Q}
\newcommand{\N}{\mathbb N}

\newcommand{\norm}[1]{\left\vert #1 \right \vert}	
\newcommand{\Norm}[1]{\left\Vert #1 \right \Vert}
\renewcommand{\span}{\operatorname{span}}
\begin{document}

\thispagestyle{empty}
\centerline{\Large Math 675 Potential Final Exam Problems}
\vspace{.25in}

The exam will consist of 20 problems, separated into definitions, examples, and proofs. Here are some potential questions. 

The list of questions may be modified before the actual exam is compiled.  You should think about what additional problems might be added.

\subsection*{Definitions}
\begin{enumerate}[1.]
\item \textbf{Metric}: single-valued, nonnegative, real valued function with the following properties
\begin{enumerate}
\item $\rho(x,y) = 0$ if and only if $x = y$
\item Symmetry: $\rho(x,y) = \rho(y,x)$
\item Triangle inequality: $\rho(x,z) \leq \rho(x,y) + \rho(y,z)$
\end{enumerate}
\item \textbf{Closure}: The set of all contact points of a set.
\item \textbf{Vector space}: A nonempty set that satisfies the following 3 axioms:
    \begin{enumerate}
        \item Any two elements $x,y \in L$ uniquely determine a third element $x+y \in L$ such that:
        \begin{enumerate}
            \item $x+y = y+x$
            \item $(x+y)+z = x+(y+z)$
            \item $\exists \; 0 \in L$ such that $x+0=x \; \forall x \in L$
            \item $\forall x \in L$, $\exists -x$ such that $x + (-x) = 0$
        \end{enumerate}
        \item Any number $a$ and any element $x \in L$ uniquely determine $ax \in L$ such that
        \begin{enumerate}
            \item $\alpha(\beta x) = (\alpha\beta)x$
            \item $1 x = x$
        \end{enumerate}
        \item Distributive properties:
        \begin{enumerate}
            \item $(\alpha + \beta)x = \alpha x + \beta x$
            \item $\alpha(x + y) = \alpha x + \alpha y$
        \end{enumerate}
    \end{enumerate}
\item \textbf{Contact point}: A point $x \in R$ is a contact point of a set $M \subset R$ if every neighborhood of $x$ contains at least one point of $M$.
\item \textbf{Cauchy}: Given any $\epsilon > 0, \; \exists N_\epsilon$ such that $\rho(x_n, x_m) < \epsilon \; \forall m, n > N_\epsilon$
\item \textbf{Complete}: A metric space is complete iff every Cauchy sequence of points in $M$ converges in $M$
\item \textbf{Sequentially compact}: Every bounded infinite set has a limit point. Every infinite sequence has a convergent subsequence.
\item \textbf{Compact}: Every open cover of $T$ has a finite subcover
\item \textbf{Separable}: A metric space is separable if it has a countable everywhere dense subset.
\item \textbf{Open}: A set that doesn't contain any of its boundary points
\item \textbf{Closed}: A set is closed if it contains all of its limit points
\item \textbf{Continuous}: A function $f$ is continuous at a point $c$ in its domain if, for any neighborhood $N_1(f(c))$ there is a neighborhood $N_2(c)$ such that $f(x) \in N_1(f(c))$ whenever $x \in N_2(c)$.
\item \textbf{Homeomorphism}: If there exists a one-to-one mapping $f$ where $f$ and $f^{-1}$ are both continuous.
\item \textbf{Isometry}: A one-to-one mapping of one metric space $R = (X,\rho)$ onto another $R^\prime = (Y,\rho^\prime)$ if $\rho \left(x_1, x _2\right) = \rho^\prime \left( f \left( x_1 \right) , f \left( x _ 2 \right) \right) \forall x_1, x_2 \in R$
\item \textbf{Completion}: Given a metric space $R$ with closure $[R]$, $R^*$ is a completion of $R$ if $R \subset R^*$ and $[R] = R^*$
\item \textbf{Contraction}: $A$ is a contraction on $R$ if $\exists \alpha < 1$ such that $\rho(Ax, Ay) \leq \alpha\rho(x,y) \; \forall x,y \in R$
\item \textbf{Linearly independent}: The elements $x, y, \ldots, w$ of a vector space are linearly independent iff $\alpha x + \beta y + \ldots + \gamma w = 0$ implies $\alpha = \beta = \ldots = \gamma = 0$
\item \textbf{Basis}: Any set of n linearly independent elements of an n-dimensional space.
\item \textbf{Span}: The intersection of all linear subspaces w hich each contain every vector in that set.
\item \textbf{Linear functional} An additive and homogenous functional:
    \begin{enumerate}
        \item $f(x+y) = f(x) + f(y) \; \forall x, y \in L$
        \item $f(\alpha x) = \alpha f(x)$ for every number $\alpha$
    \end{enumerate}
\item \textbf{Null space}: The set of all elements $v$ of $V$ for which $L(v) = 0$.
\item \textbf{Hyperplane}: Every class in $L$ generated by any subspace of codimension 1.
\item \textbf{Codimension}: The dimension of the factor space $L|L^\prime$ where $L$ is any linear space and $L^\prime$ is a subspace of $L$. For finite dimensions, the codimension of $W$ in $V$: $\operatorname { codim } ( W ) = \operatorname { dim } ( V ) - \operatorname { dim } ( W )$
\item \textbf{Convex}: A set $M \subset L$ is convex if whenever it contains two points $x$ and $y$, it also contains the segment joining $x$ and $y$.
\item \textbf{Convex body}: A convex set is a convex body if its interior is nonempty.
\item \textbf{Convex functional}: A functional $p$ defined on a linear space $L$ where:
    \begin{enumerate}
        \item $p(x) > 0 \; \forall x \in L$
        \item $p(\alpha x) = \alpha p(x) \; \forall x \in L, a \geq 0$
        \item $p(x+y) \leq p(x) + p(y) \; \forall x,y \in L$
    \end{enumerate}
\item \textbf{Quotient space}: The set of classes where $x,y \in L^\prime$ if $x-y \in L^\prime$. Also known as the factor space.
\item \textbf{Minkowski functional}: Let $E$ be a convex body whose interior contains the point 0.
    $p_E(x) = \inf\{r: \frac{x}{r} \in E, r > 0\}$
\item \textbf{Subspace}: If $M \subset X$ and $R^* = (M, \rho)$ is a metric space, then $R^*$ is a subspace of the metric space $R = (X, \rho)$. A nonempty subset $L^\prime$ of a linear space $L$, where $x, y \in L^\prime \Rightarrow \alpha x + \beta y \in L^\prime$ for arbitrary $\alpha, \beta$.
\item $d_p$: $d _ { p } ( x , y ) : = \left\{ \begin{array} { l l } { \left[ \sum\limits_ { k = 1 } ^ { n } \left| x _ { k } - y _ { k } \right| ^ { p } \right] ^ { 1 / p } , } & { 1 \leq p < \infty } \\ { m a x _ { k = 1 , \ldots , n } \left| x _ { k } - y _ { k } \right| , } & { p = \infty } \end{array} \right.$
\item $C_p[a,b]$: The space of continuous functions that have continuous first p derivatives over the inverval $[a,b]$.
\item $\ell_p$: All sequences of of scalars such that $\| x \| _ { p } = \left( \sum\limits _ { i = 1 } ^ { k } \left| x _ { i } \right| ^ { p } \right) ^ { 1 / p } < \infty$ for $1 \leq p < \infty$
\item Norm
\item Inner product
\item Bessel's inequality
\item Parseval's identity
\item Fourier coefficient
\item Hilbert space
\item Banach space
\item H\"older's inequality
\item Cauchy's inequality
\item Orthogonal complement
\item Riesz-Fisher Theorem
\item Continuous linear functional
\item Dual space
\item Dual operator
\item Spectrum
\item Adjoint
\item Eigenvalue
\item Eigenvector
\item Completely continuous
\item Compact operator
\item Self-adjoint
\item Weierstrass approximation theorem
\end{enumerate}
\newpage
\subsection*{Proofs}
\begin{enumerate}[1.]
\item \textbf{For $a<b$ in $\R$, the closure of $(a,b)$ is $[a,b]$.}
\begin{flushleft}
    $a,b$ are limit points for $(a,b)$. Given $\delta > 0$, the intervals $(a-\delta, a+\delta)$ and $(b-\delta, b+\delta)$ contains infinitely many points of $(a,b)\setminus \{a\}$ and $(a,b)\setminus \{b\}$, respectively. Therefore $a$ and $b$ are limit points of $(a,b)$ and so $\overline{(a, b)} \subset [a,b]$. Let $x \not\in (a,b)$, with $x > b$. If we set $\delta < |x-b|$, then $( x - \delta , x + \delta ) \cap ( a , b ) \backslash \{ x \} = \emptyset$. Therefore $[a,b] \subset \overline{(a,b)}$ and combining this with the first part gives us: $\overline{(a,b)} = [a,b]$ $\Box$
\end{flushleft}
\item \textbf{In $\R^n$, the metric $d_\infty$ is the limit of the functions $d_p$ as $p\rightarrow \infty$.}
\begin{flushleft}
    If the maximum value of $|x_i - y_i|$ occurs when $i = k$, let us set $\max |x_i - y_i| = |x_k - y_k|$.
    $$ |x_k-y_k|^p \leq \sum_{i=1}^n |x_i-y_i|^p \leq n \cdot |x_k-y_k|^p $$
    
    $$ \lim_{p\to\infty}\left(|x_k-y_k|^p\right)^{1/p} \leq \lim_{p\to\infty}\left(\sum_{i=1}^n |x_i-y_i|^p\right)^{1/p} \leq \lim_{p\to\infty}\left(n \cdot |x_k-y_k|^p\right)^{1/p}$$
    
    $$ |x_k-y_k| \leq \lim_{p\to\infty}\left(\sum_{i=1}^n |x_i-y_i|^p\right)^{1/p} \leq \lim_{p\to\infty}(n^{1/p}) \cdot |x_k-y_k| = |x_k-y_k|$$
    Using the squeeze theorem for sequences, since the sum is bounded on both sides by $|x_k-y_k|$, we have proven that the limit is equal to $|x_k-y_k| = \max |x_i - y_i|$. Therefore,
    
    $$\max |x_i-y_i|=\lim_{p\rightarrow \infty}\left(\sum_{i=1}^n |x_i-y_i|^p\right)^{1/p} \Box$$
\end{flushleft}
\item \textbf{The spaces $C_\infty[a,b]$ and $C_\infty[c,d]$ are isometric for any $a<b$ and $c<d$.}
\begin{flushleft}
    We need to find $T(x)$ such that:
    $$ d_\infty(f,g) = \max_{x\in [a,b]} |f(x) - g(x)| = \max_{x\in [c,d]} |T(f(x)) - T(g(x))| = d_\infty(T(f),T(g)) $$
    Let $T$ be the map between $C_\infty[c,d]$ and $C_\infty[a,b]$ defined as follows:
    $$ (T\circ f)(x) = f\left(\frac{b-a}{d-c} (x-c) + a \right) $$
    Since $x \in [c,d]$ if and only if $\frac{b-a}{d-c} (x-c) + a \in [a,b]$, we can say that $T\circ f$ is well-defined over $[c,d]$ and is a composition of well-defined functions. $T(ef + g) = eT(f) + T(g)$, so $T$ is a linear map.
    $$ \| T\circ f\|_\infty = \max_{x\in [c,d]} |(T\circ f)(x)| = \max_{x\in [c,d]}\left|f\left(\frac{b-a}{d-c} (x-c) + a \right)\right| =  \left|f\left(\frac{b-a}{d-c} (y-c) + a \right)\right| = \max_{x\in [a,b]} |f(x)| = \| f\|_\infty $$
    
    This equation is true for some $y \in [c,d]$ as $T\circ f$ is onto and $y$ corresponds to the maximum value
    $$ d_\infty(f,g) = \|f-g\|_\infty = \|T(f-g)\| = \|T(f) - T(g)\| = d_\infty(T(f),T(g)) $$
    $$ \Rightarrow d_\infty(f,g) = d_\infty(T(f),T(g)) \Leftrightarrow T \text{ is an isometry between } C_\infty[a,b] \text{ and } C_\infty[c,d] \, \Box$$. 
\end{flushleft}
\item \textbf{A point $x\in X$ is a limit point of $A$ if and only if for every $\epsilon>0$ the ball $B_\epsilon(x)$ contains a point of $A$ not equal to $x$.}
\begin{flushleft}
    From the text, a limit point is defined as follows:  A point $x \in X$ is called a limit point of a set $A \subset X$ if every neighborhood of $x$ contains infinitely many points of $A$. \newline $(\Rightarrow)$ Assume: $x \in X$ is a limit point of $A$ and need to show: $\forall \epsilon > 0$, $B_\epsilon(x)$ contains a point of $A$ not equal to $x$. By the definition of a limit point, $\forall \epsilon > 0$, $B_\epsilon(x)$ contains infinitely many points of $A$. Removing $x$ from this set still leaves infinitely many points of $A$. Therefore $B_\epsilon(x)$ contains a point of $A$ not equal to $x$.
    \newline $(\Leftarrow)$ Assume: $\forall \epsilon > 0$, $B_\epsilon(x)$ contains a point of $A$ not equal to $x$ and need to show: $x \in X$ is a limit point of $A \Leftrightarrow$ every neighborhood of $x$ contains infinitely many points of $A$. We shall prove this theorem via the proof of the following equivalent contrapositive statement: some neighborhood of $x$ contains finitely many points of $A \Rightarrow \exists \epsilon > 0$, $B_\epsilon(x)$ does not contains a point of $A$ not equal to $x$. Proof: Assume $\exists \epsilon > 0$ such that $B_\epsilon(x) \cap A$ contains finitely many points. Then, let $E = (B_\epsilon(x) \cap A)\setminus \{x\}$, which also has finitely many points. If $E = \emptyset$, we have shown our desired conclusion of $B_\epsilon(x)$ containing no point of $A$ not equal to $x$. If $E \not= \emptyset$, let $E = \{e_1, \ldots, e_n\}$ and let $D = \{d(e_1,x), \ldots, d(e_n, x)\}$, which will also be finite. If we take $\epsilon^\prime = \frac{1}{2} \min\{d(e_1,x), \ldots, d(e_n, x)\}$, then $B_{\epsilon^\prime}(x) \cap A = \emptyset$. $\epsilon^\prime < \epsilon \Rightarrow B_{\epsilon^\prime}(x) \subset B_\epsilon(x)$. Therefore, $B_{\epsilon^\prime}(x) \cap A = (B_{\epsilon^\prime}(x) \cap B_\epsilon(x))\cap A = B_{\epsilon^\prime}(x) \cap (B_\epsilon(x)\cap A) = B_{\epsilon^\prime}(x) \cap E = \emptyset$. So we have found a neighborhood of $x$ that contains no points of $A$, except for possibly $x$, so $\exists \epsilon > 0$, $B_\epsilon(x)$ does not contains a point of $A$ not equal to $x$, namely $\epsilon^\prime$. So we have proven via contrapositive, that if for every $\epsilon>0$ the ball $B_\epsilon(x)$ contains a point of $A$ not equal to $x$, then $x\in X$ is a limit point of $A$. Together with the first half, we have proven that $x\in X$ is a limit point of $A$ if and only if for every $\epsilon>0$ the ball $B_\epsilon(x)$ contains a point of $A$ not equal to $x$. $\Box$
\end{flushleft}
\item \textbf{The closure of $A$ equals the union of the set of isolated points $A$ with the set of limit points of $A$.}
\begin{flushleft}
    Let $\bar{A}$ denote the closure of $A$, let $A_{limit}$ denote the limit points of $A$, and let $A_{isolated}$ denote the isolated points of $A$. We need to show that $\bar{A} = A_{limit} \cup A_{isolated}$. We shall show that by proving $(A_{limit} \cup A_{isolated}) \subset \bar{A}$ and $\bar{A} \subset (A_{limit} \cup A_{isolated})$. \newline Lemma 1: $A_{limit} \subset \bar{A}$ is true by the following proof by contradiction. Let $x \in A_{limit}$ be a limit point and let us suppose that $x \not\in \bar{A}$. The point $x$ would be in the complement of $\bar{A}$ and so it would have a neighborhood $B_\epsilon(x)$ disjoint from $\bar{A}$, which would contradict the definition of a limit point. Therefore, $x \in A_{limit} \Rightarrow x \in \bar{A}$. This proves our lemma that $A_{limit} \subset \bar{A}$.
    \newline Every isolated point of set $A$ is contained in set $A$. Therefore, $x \in A_{isolated} \Rightarrow x \in A \Rightarrow x \in \bar{A}$, or equivalently $A_{isolated} \subset A \subset \bar{A} \Rightarrow A_{isolated} \subset \bar{A}$. Combining the previous two results gives us: $A_{limit} \subset \bar{A} \wedge A_{isolated} \subset \bar{A} \Rightarrow (A_{limit} \cup A_{isolated}) \subset \bar{A}$
    \newline $\bar{A} \subset (A_{limit} \cup A_{isolated})$ is equivalent to: $x \in \bar{A} \Rightarrow (x \in A_{limit} \vee x \in A_{isolated})$. Let us take $x \in \bar{A}$, and if $x$ is a limit point, we are done since $x \in A_{limit}$, which satisfies the desired condition. So we must show that if $x$ is not a limit point, that it will be an isolated point of the set $A$. If $x$ is not a limit point, $\exists B_\epsilon(x)$ with finitely many points of $A$, $\{a_1, \ldots, a_n\}$. This this is a finite set, so is the set of distances from each point to $x$ namely, $\{d(a_1,x), \ldots, d(a_n, x)\}$. If we take $\epsilon^\prime = \frac{1}{2} \min\{d(a_1,x), \ldots, d(a_n, x)\}$, then $B_{\epsilon^\prime}(x) \cap A = \{x\}$, since $\epsilon^\prime$ is less than the distance to any of the other finite elements of $A$. This by definition means that $x \in A_{isolated}$. We have thus shown that $x \in \bar{A} \Rightarrow (x \in A_{limit} \vee x \in A_{isolated})$, or equivalently, $\bar{A} \subset (A_{limit} \cup A_{isolated})$. Combining our previous results gives us: $(A_{limit} \cup A_{isolated}) \subset \bar{A} \wedge \bar{A} \subset (A_{limit} \cup A_{isolated}) \Leftrightarrow \bar{A} = A_{limit} \cup A_{isolated}$, as desired. $\Box$
\end{flushleft}
\item \textbf{What is the closure of $\Q\subset \R$? Prove that you are correct.}
\begin{flushleft}
    The closure of $\Q\subset \R$ is $\bar{\Q} = \R$. For any $p \in \R$, and $\forall \epsilon > 0$, $\exists q \in \Q$ such that $q \in (p-\epsilon,p+\epsilon) = B_\epsilon(p)$. This is true since $p-\epsilon$ and $p+\epsilon$ are real numbers and since $\Q$ is dense in $\R$, we can guarantee that such a $q \in \Q$ exists. Therefore, $p$ is a limit point of $\Q \, \forall p \in \R$ and so the set of limit points of $\Q$ is $\R$, which we denote as: $\Q_{limit} = \R$. By the definition of closure, $\bar{E} = E \cup E_{limit}$. Applied to the rationals, $\bar{\Q} = \Q \cup \Q_{limit} = \Q \cup \R = \R$. This proves that $\bar{\Q} = \R$, as desired. $\Box$
\end{flushleft}
\item \textbf{The following are equivalent for a function $f: X\rightarrow Y$ between two metric spaces:}
\begin{enumerate}[(i)]
\item $f$ is continuous,
\item $f^{-1}(A)$ is an open set for every open set $A\subset Y$,
\item $f^{-1}(B)$ is a closed set for every closed set $B\subset Y$.
\end{enumerate}
\begin{flushleft}
    $(i) \Rightarrow (ii)$: We are given that $A \subset Y$ is an open set and that $f$ is continuous. For a given $a \in f^{-1}(A)$, we know that $f(a) \in A$. Given that $A$ is open, $\exists \epsilon > 0$ such that $B_\epsilon(f(a)) \subset A$. Since $f$ is continuous, $\exists \delta > 0$ such that $d(x,a) < \delta \Rightarrow d(f(x),f(a)) < \epsilon$. Analogously, $x \in B_\delta(a) \Rightarrow f(x) \in B_\epsilon(f(a))$. Therefore, $f(x) \in A \Rightarrow x \in f^{-1}(A)$. Since $x \in B_\delta(a) \Rightarrow x \in f^{-1}(A)$, $B_\delta(a) \subset f^{-1}(A)$, and so $f^{-1}(A)$ is an open set. \newline
$(ii) \Rightarrow (i)$: We are given that $f^{-1}(A)$ is an open set for every open set $A\subset Y$. As before, for a given $a \in f^{-1}(A)$, we know that $f(a) \in A$. Given that $A$ is open, $\exists \epsilon > 0$ such that $B_\epsilon(f(a))$ is open. Since $f(a) \in B_\epsilon(f(a))$, $a \in f^{-1}(B_\epsilon(f(a)))$. Due to the assumption from $(ii)$, $B_\epsilon(f(a))$ is open $\Rightarrow f^{-1}(B_\epsilon(f(a)))$ is open. Given that $ f^{-1}(B_\epsilon(f(a)))$ is open, $\exists \delta > 0$ such that $B_\delta(a) \subset f^{-1}(B_\epsilon(f(a)))$. Equivalently, $x \in B_\delta(a) \Rightarrow x \in f^{-1}(B_\epsilon(f(a))) \Rightarrow f(x) \in B_\epsilon(f(a))$. In other words, $d(x,a) < \delta \Rightarrow d(f(x),f(a)) < \epsilon$. Since we can find $\delta > 0$ for any $\epsilon > 0$, this proves that $f$ is continuous. \newline
$(ii) \Rightarrow (iii)$: A subset $B$ of a metric space $Y$ is called closed if its complement is open in $Y$. For a given closed set $B \subset Y$, we are to show that $f^{-1}(B)$ is closed. If $B$ is closed, it's complement $Y \setminus B$ must be open. If $Y \setminus B$ is open, then by $(ii)$, $f^{-1}(Y \setminus B)$ is also open. \newline
Lemma 1: $f^{-1}(Y \setminus B) = X \setminus f^{-1}(B)$. \newline Proof: $a \in f^{-1}(Y \setminus B) \Leftrightarrow f(a) \in Y \setminus B \Leftrightarrow f(a) \not\in B \Leftrightarrow a \not\in f^{-1}(B) \Leftrightarrow a \in X \setminus f^{-1}(B)$. Therefore, $[f^{-1}(Y \setminus B) \subset X \setminus f^{-1}(B)$ and $X \setminus f^{-1}(B) \subset f^{-1}(Y \setminus B)] \Leftrightarrow f^{-1}(Y \setminus B) = X \setminus f^{-1}(B)$, as desired. \newline
Since $f^{-1}(Y \setminus B)$ is open, $X \setminus f^{-1}(B)$ is also open. The complement of this set, $f^{-1}(B)$, must therefore be closed. \newline
$(iii) \Rightarrow (ii)$: We are given that $f^{-1}(B)$ is a closed set for every closed set $B\subset Y$. If $A$ is an open set, its complement $Y\setminus A$ is closed and so $f^{-1}(Y\setminus A)$ is closed. As seen above, inverse images commute with complements and so, $X \setminus (f^{-1}(Y\setminus A)) = f^{-1}(Y\setminus A)$. Therefore, the complement of $f^{-1}(Y\setminus A)$ is closed, and so $f^{-1}(A)$ is open. Therefore we have shown that if $A$ is an open set, $f^{-1}(A)$ is open.\newline
We have proven equivalence by proving that: $(i) \Leftrightarrow (ii) \Leftrightarrow (iii)$. $\Box$
\end{flushleft}
\item  \textbf{The set of invertible 2-by-2 matrices is open. (Interpret the space of all 2-by-2 matrices as $\R^4$ with the usual topology.) You may use your favorite definition of continuity.}
\begin{flushleft}
    Let $A$ denotes the set of $M_{2\times2}$ that are also invertible.
    Let $det: A \rightarrow \R\setminus\{0\}$ be the determinant function for invertible 2-by-2 matrices.
    \[
    |A| = 
    \begin{vmatrix}
        a_{11} & a_{12} \\
        a_{21} & a_{22} \\
    \end{vmatrix}
    = a_{11}\cdot a_{22} - a_{12}\cdot a_{21}
    \]
    Since $det$ is a quadratic polynomial, it is a continuous function. Using 1(ii), we know that $f^{-1}(A)$ is an open set for every open set $A\subset Y$ if $f$ is a continuous function. $\R\setminus\{0\}$ is an open set, as it is a finite union of open sets, $(-\infty, 0) \cup (0, \infty)$. Since $\R\setminus\{0\}$ is an open set, $det^{-1}(\R\setminus\{0\}) = A$ must also be an open set. $\Box$
    \end{flushleft}
\item \textbf{The set of determinant-one 2-by-2 matrices is closed. You may use your favorite definition of continuity.}
\begin{flushleft}
    Let $B$ denote the set of $M_{2\times2}$ that have determinant $= 1$.
    Let $det: B \rightarrow \{1\}$ be the determinant function for determinant-one 2-by-2 matrices. As before, we know that the determinant function is a continuous quadratic and so we can utilize $(i) \Rightarrow (iii)$ from proof 7 to show that $B$ is a closed set. The image of the $det$ function is the closed set $\{1\}$ and so the pre-image $det^{-1}(\{1\}) = B$, must also be a closed set. $\Box$
    \end{flushleft}
\item \textbf{Let $A$ be a linearly independent set in a metric vector space $V$. Assuming that $V$ is finite-dimensional, prove that the set $\R A=\{\lambda a : \lambda \in \R, a\in A\}$ is closed.}
\begin{flushleft}
    Let A be the linearly independent set $\{a_1, \ldots, a_n \}$, where $n < \infty$. We shall prove that the set $\R A=\{\lambda a : \lambda \in \R, a\in A\}$ is closed by showing that it is not open, regardless of the neighborhood of points considered. Without loss of generality, for any given $\lambda a_1 \in \R A$, let us take a ball $B_\epsilon(\lambda a_1)$. By definition, $\lambda a_1 + \epsilon a_2$ is within $B_\epsilon(\lambda a_1)$ but is not within $\R A$ as it does not contain any linear combination of independent vectors. Alternatively, we can also say that $\R A$ contains all of its limit points as every neighborhood contains infinitely many points in $\R A$. For a given $B_\epsilon(\lambda a_1)$, there exists infinitely many $\gamma \in \R$ such that $\gamma \lambda a_1 < \epsilon \lambda a_1$. Since $\R A$ contains all of its limit points, it is closed. $\Box$
    \end{flushleft}
\item \textbf{Suppose two metric spaces $X$ and $Y$ are bi-Lipschitz equivalent. Prove that that $X$ is complete if and only if $Y$ is complete.}
\begin{flushleft}
    We can prove that $Y$ is complete if $X$ is complete and since the relation is bi-directional, the converse is proven simultaneously without loss of generality. Let $\{y_k\}$ be a Cauchy sequence in $Y$, which means that given any $\epsilon_1 > 0$, $\exists N \in \N$ such that if $n, m \geq N$, $d_Y(y_n, y_m) < \epsilon_1$. Since $X$ and $Y$ are bi-Lipschitz equivalent, we have: $\frac{1}{K}d_X(x_1, x_2) \leq d_Y(f(x_1), f(x_2)) \leq Kd_X(x_1, x_2)$ with a continuous surjective function $f$. Equivalently, this can be stated as: $d_X(f^{-1}(y_1), f^{-1}(y_2)) \leq Kd_Y(y_1, y_2)$. For the Cauchy sequence $\{y_k\}$ we get: $d_X(f^{-1}(y_n), f^{-1}(y_m)) \leq Kd_Y(y_n, y_n) \leq K\epsilon_1 = \epsilon_1$. Since $d_X(f^{-1}(y_n), f^{-1}(y_m)) \leq \epsilon_2$ for any given $\epsilon_2 > 0$, $\{f^{-1}(y_k)\}$ forms a Cauchy sequence in $X$ and converges to some $x \in X$ as $X$ is complete by assumption. $f(f^{-1}(y_k)) = y_k$ converges to $f(x) = y \in Y$ since $f$ is a continuous function. Since we have shown that any arbitrary Cauchy sequence $\{y_k\}$ converges to a point $y \in Y$, $Y$ is proven to be complete. $\Box$
\end{flushleft}
\item \textbf{The image of a compact set under a continuous function is compact.}
\begin{flushleft}
    Consider an open cover of $f(X)$. Then $f(X) \subset \bigcup_{\alpha\in A} V_\alpha$ where each $V_\alpha$ is open in $Y$. $X \subset f^{-1}(f(X)) \subset f^{-1}\left( \bigcup_{\alpha\in A} V_\alpha \right) = \bigcup_{\alpha\in A} f^{-1}(V_\alpha)$. Therefore $\bigcup_{\alpha\in A} f^{-1}(V_\alpha)$ is an open cover of $X$. Since $X$ is compact, we can choose a finite subcover $X \subset \bigcup_{i=1}^n f^{-1}(V_i)$. Then $f(X) \subset f\left(\bigcup_{i=1}^n f^{-1}(V_i)\right) = \bigcup_{i=1}^n f\left(f^{-1}(V_i)\right) \subset \bigcup_{i=1}^n V_i$. Therefore $f(X)$ is compact.
\end{flushleft}
\item \textbf{Let $f: X\rightarrow X$ be a contraction, and $Y\supset X$ a completion of $X$. Prove that $f$ extends to a contraction of $Y$. Use this to conclude that there is no homeomorphism of $(0,1)$ that is also a contraction.}
\begin{flushleft}
    https://math.stackexchange.com/questions/310748/suppose-f-m-to-m-is-a-contraction-but-m-is-not-necessarily-complete
\end{flushleft}
\vspace{1in}
\item  \textbf{The contraction theorem requires a mapping that is $L$-Lipschitz for $L<1$. Prove that the result is sharp. That is, allowing $L=1$ in the theorem would make it false.}
\begin{flushleft}
    If $L=1$, we have $d(F(x),F(y)) \leq d(x,y)$ as the 1-Lipschitz condition with $F$ as the contraction mapping and $x,y \in X$, a complete metric space. If we iterate the map $n$ times, we get: $d(F^n(x),F^n(y)) \leq d(x,y)$. We shall show, using proof by contradiction, that there does not exist a unique, converging fixed point, or equivalently, $d(F^n(x),F^n(y)) \not= 0$ as $n \rightarrow \infty$. Suppose a converging fixed point existed, then $(F^n(x))_{n\in\N}$ would be a Cauchy sequence. This means that for any given $\epsilon > 0$ there exists an integer $N$ such that $m>n>N$ implies that $d(F^m(x),F^n(x)) < \epsilon$. Using the definition of the map and the 1-Lipschitz condition, we have: $d \left(F^{m}(x),F^{n}(x)\right) \leq \sum\limits_{k=0}^{m-n-1} d \left(F^{n+k+1}(x), F^{n+k}(x) \right) \leq \sum\limits_{k=0}^{m-n-1} d(F(x),(x))$. Unless $F(x) = x$, the sum cannot be less than any arbitrary $\epsilon$ and so the sequence $(F^n(x))_{n\in\N}$ does not converge and so is not Cauchy. If $F(x) = x$, then every point would be a fixed point and it would not be unique. If the sequence is not Cauchy, there is not a unique, converging fixed point in the given metric space. Therefore allowing $L=1$ makes the contraction theorem false. $\Box$
\end{flushleft}
\item \textbf{A closed subset of a compact space is compact.}
\begin{flushleft}
    Let $A$ be a closed subset of a compact space $X$ and let $\mathcal{U} = \{U_i\}$ be an open cover of $A$. Let us define $\mathcal{V} = \{U_i\} \cup \{X\setminus A\}$. Since $A$ is closed, $X\setminus A$ is open and so is its union with the open cover $\mathcal{U}$. Since the union contains both $A$ and $X\setminus A$, it is an open cover of the compact space $X$. Since $\mathcal{V}$ is an open cover of a compact space, it has a finite subcover denoted as $\{U_{i_k}\}_{k=1}^n \cup \{X\setminus A\}$. Since the finite subcover covers the entire space, $\{U_{i_k}\}_{k=1}^n$ must cover the closed subset $A$. Since $A$ is covered by a finite subcover, $A$ must be compact. $\Box$
\end{flushleft}
\item \textbf{Suppose $f, f_1, f_2$ are linear functionals such that $f(x)=0$ if $f_1(x)=0$ and $f_2(x)=0$. Prove that there are constants $a_1$ and $a_2$ such that $f=a_1 f_1+a_2f_2$.}
\begin{flushleft}
    Applying mathematical induction, we observe that in ker $f _ { n }$ we have $n$
linear functionals $f , f _ { 1 } , f _ { 2 } , \ldots , f _ { n - 1 }$ having the property that $f ( x ) = 0 ,$ whenever $f _ { 1 } ( x ) = f _ { 2 } ( x ) = \cdots = f _ { n - 1 } ( x ) = 0$ if $x \in$ ker $f _ { n } .$ Consequently, there exists $\lambda _ { 1 } , \lambda _ { 2 } , \ldots , \lambda _ { n } \in \mathbb { R } ,$ such that $f ( x ) = \sum _ { i = 1 } ^ { n - 1 } \lambda _ { i } f _ { i } ( x ) , x \in \operatorname { ker } f _ { n }$
Now, we observe that $f _ { n }$ and $f - \sum _ { i = 1 } ^ { n - 1 } \lambda _ { i } f _ { i }$ are two linear functionals having the same kernels. Since the kernel of a nontrivial linear functional is a homogenous hyperplane and for every homogenous hyperplane there exists a functional uniquely determined up to a nonzero multiplicative constant with the kernel, there exists $\lambda _ { n } \in \mathbb { R } ,$ such that $f - \sum _ { i = 1 } ^ { n - 1 } f _ { i } =$
$\lambda _ { n } f _ { n } ,$ as claimed.
\end{flushleft}
\item \textbf{If $X$ is compact and $f: X\rightarrow \R$ is continuous, then $f$ attains a maximum value.}
\begin{flushleft}
    Since $X$ is compact and $f$ is continuous, $f(X)$ is a compact subset of $\R$ and therefore closed and bounded. Since $f(X)$ is bounded, it has a supremum and since it is closed, $\sup f(X) \in f(X)$. Therefore, there is $x_0 \in X$ such that $f(x_0) = \sup f(X)$ so $f(X)$ attains a maximum value at $x_0$. $\Box$
\end{flushleft}
\item \textbf{If $A, B\subset X$ are compact subsets of a metric space $X$, then $A\cup B$ is compact.}
\begin{flushleft}
    Let $\mathcal{U} = \{U_\gamma\}$ be an arbitrary open cover of $A \cup B$. By the definition of open cover, this means that $A \cup B \subset \mathcal{U}$ and so both $A \subset \mathcal{U}$ and $B \subset \mathcal{U}$. Therefore $\mathcal{U}$ is an open cover of both $A$ and $B$. Since $A$ and $B$ are compact subsets, it implies every open cover of $A$ and $B$ has a finite subcover. Denote the finite subcover of $A$ as $\mathcal{U}_A = \bigcup\limits_{i=1}^n \mathcal{U}_{\alpha_i}$ and the finite subcover of $B$ as $\mathcal{U}_B = \bigcup\limits_{i=1}^m \mathcal{U}_{\beta_i}$. Since the union of two finite sets is finite, $\mathcal{U}_A \cup \mathcal{U}_B$ is also finite and specifically, is a finite subcover for $A \cup B$. Since there exists a finite subcover for any arbitrary open cover for $A \cup B$, we have shown that $A \cup B$ is compact. $\Box$
\end{flushleft}
\item Suppose $A,B\subset V$ are separated by a linear functional $f$. If $A$ is a convex body, prove that there are points $a\in A$ and $b\in B$ such that $f(a)\neq f(b)$.
\item Let $\Norm{\cdot}$ be the norm in $\R^2$ for which the unit circle is a regular hexagon with side length 1. Prove that $\Norm{\cdot}$ is not induced by an inner product. (Hint: the parallelogram law states  $2(a^2+b^2)=(a-b)^2+(a+b)^2$).
\item Given a Banach space $V$, let $\{B_n\}$ be a nested sequence of closed spheres in $V$. Prove that $\cap_n B_n$ is nonempty. (Note that the radius is not assumed to go to zero, and the centers are not assumed to be the same.)
\item Let $\{e_i\}$ be an orthonormal basis for a Hilbert space $H$. Take $x_n=e_{2n}$ and $y_n=e_{2n}+\frac{e_{2n+1}}{n+1}$, and set $M=\span(\{x_n\})$, $N=\span(\{y_n\})$. Show that $M+N$ is not closed, even though $M$ and $N$ are both closed.
\item Prove that the functional $F(f)=\int_a^b f(t) \cos(t)dt$ is a continuous functional on $C_\infty[a,b]$. What could we replace $\cos(t)$ with?
\item Prove that both of the following is a subspace of $\ell^2$: the set of all $(x_i)$ such that $x_1=x_2$.
\item Prove that both of the following is a subspace of $\ell^2$: the set of all $(x_i)$ such that  $x_k=0$ for all even $k$.
\item Let $V$ be a normed real vector space and $F: V\rightarrow \R$ a linear functional. Prove that $F$ is continuous if and only if $N(F)$ is closed. (Hint: The forward direction requires no work at all; for the backward direction, assume $F$ is not bounded and show that $N(F)$ is not closed by perturbing a non-zero point.)
\item Prove that  following functional is continuous on $C_\infty[0,1]$ and compute its norm: $f(x)=ax(0)+bx(1)$.
\item Prove that  following functional is continuous on $C_\infty[0,1]$ and compute its norm: $g(x)=\int_0^{1/2}x(t)dt - \int_{1/2}^1 x(t)dt$.
\item Prove that if $p<q<\infty$ and $f$ is a linear functional on $C_p[0,1]$ then it is also continuous on $C_q[0,1]$.
\item Let $V_0$ be a normed real vector space and $V$ its completion. Prove that $V_0^*$ and $V^*$ are  isomorphic  Banach spaces.
\item Recall that a sequence $v_i$ in a normed space $V$ converges weakly to $v$ if $f(v_i)$ converges to $f(v)$ for each linear functional $f\in V^*$. Prove that the standard basis vectors $e_i$ in $\ell_2$ weakly converge to the zero vector.
\item Prove that an infinite-dimensional Banach space has uncountable algebraic dimension.
\item Prove that every element of $C_2[a,b]$ is of the form $\sum_1^\infty a_ip_i(x)$ for $a_i\in \R$ and $p_i$ a polynomial of degree $i$.
\item A linear functional is continuous if and only if it is bounded.
\item Let $V\subset C_\infty[a,b]$ consist of all continuously differentiable functions. Prove that the differentiation operator $D:V\rightarrow C_\infty[a,b]$ is not continuous.
\end{enumerate}
\newpage
\subsection*{Examples}
\begin{enumerate}[1.]
\item \textbf{A bounded set that is not compact.}
\begin{flushleft}
    Let $X = \{ \frac{1}{n}: n \in \N\}$ and let $\rho(x,y) = 0$ if $x=y$ and 1 otherwise. $X$ is bounded, as any ball of radius greater than 1 includes the whole set. However, $X$ is not compact, since the open cover by singletons is not a finite subcover.
\end{flushleft}
\item \textbf{A metric vector space that is not separable.}
\begin{flushleft}
    The Banach space $l^\infty$ of all bounded sequences with the supremum norm.
\end{flushleft}
\vspace{.75in}
\item \textbf{A linearly independent set $A$ in a metric vector space $V$ such that the set $\R A=\{\lambda a : \lambda \in \R, a\in A\}$ is not closed.}
\begin{flushleft}
    The vector space of $l_2$ is the space of square-summable sequences $x = (x_1, x_2, \ldots)$. A linearly independet set can be constructed as follows: $(1, 0, \ldots), (0, 1, 0, \ldots), \ldots, (0, \ldots, 0, 1, 0, \ldots)$ with the 1 in the ith position. Clearly this set is linearly independent as none are a linear combination of the others. The closure of the set $\R A$ contains the following limit point $(1, 1/2, 1/3, \ldots)$, which is not in $l_2$, so it is not closed.
\end{flushleft}
\item \textbf{A union of closed sets that is not closed.}
$$\bigcup_{n=2}^\infty \left[ \frac{1}{n}, 1 - \frac{1}{n} \right] = (0,1)$$
\item \textbf{An intersection of open sets that is not open.}
$$\bigcap_{n=1}^\infty \left(-\frac{1}{n}, \frac{1}{n} \right) = \{0\}$$
\item \textbf{Homeomorphic metric spaces $X$ and $Y$ such that $X$ is complete but $Y$ isn't.}
\begin{flushleft}
    Let $X = \R$, which is a complete metric space while we let $Y = (0,1)$, an incomplete metric space. $Y$ is incomplete since there exists a Cauchy sequence in $Y$ such as $\{x_n\} = \frac{1}{n}$ but does not converge to a point in $Y$, as $\lim_{n\rightarrow\infty}x_n = 0 \not\in Y$. These spaces are still homeomorphic via the map $f:Y\rightarrow X: f(y) = \tan(\pi(y-\frac{1}{2}))$, which maps the pre-image $(0,1)$ to the image $\R$. $\Box$
\end{flushleft}
\item \textbf{A compact set $C$ and a continuous function $f$ such that $f^{-1}(C)$ is not compact.}
\begin{flushleft}
    $f(x) = 0 \; \forall x \in \R$. $C = \{0\}$ is compact but $f^{-1}(C) = \R$, which is not compact
\end{flushleft}
\item \textbf{A contraction that does not have a fixed point.}
\begin{flushleft}
    $f(x) = x/2$ on $\R\setminus\{0\}$
\end{flushleft}
\item \textbf{Let $V$ be a non-trivial vector space. Give an example of a non-trivial functional on $V^*$.}
\item \textbf{A convex set that is not a convex body.}
\begin{flushleft}
    The Hilbert cube is a convex set but not a convex body.
\end{flushleft}
\item \textbf{Two sets that cannot be separated by a linear functional.}
\begin{flushleft}
    Now we want to design a convex set $K$ not containing the origin such that the only linear functional $l$ that is non-negative on this set is 0. To this end, take the space $X$ to be the space of all real sequences with finitely many non-zero terms and let $K$ be the set of all such sequences whose last non-zero element is positive. Now, if $x \in X$, choose $y$ to be any sequence whose last non-zero element is 1 and lies beyond the last non-zero element in $x$. Then for every $\delta > 0$, both $x+\delta y$ and $-x+\delta y$ are in $K$, so $\pm l(x) + l(y) \geq 0$ with any $\delta > 0$. Therefore $l(x) = 0$ and the sets cannot be separated by a linear functional.
\end{flushleft}
\item \textbf{A Cauchy sequence in $C_2[a,b]$ that does not converge to a function in $C_2[a,b]$.}
\begin{flushleft}
    $$\phi _ { n } ( t ) = \left\{ \begin{array} { c l } { - 1 } & { \text { if } - 1 \leq t < - 1 / n } \\ { n t } & { \text { if } - 1 / n \leq t < 1 / n } \\ { 1 } & { \text { if } 1 / n \leq t \leq 1 } \end{array} \right.$$
    Assume $\phi_n$ converges to $\phi$ and let
    $$f ( t ) = \left\{ \begin{array} { c l } { - 1 } & { \text { if } - 1 \leq t < 0 } \\ { 1 } & { \text { if } 0 \leq t \leq 1 } \end{array} \right.$$
    $$\left( \int ( f ( t ) - \phi ( t ) ) ^ { 2 } d t \right) ^ { 1 / 2 } \leq \left( \int \left( f ( t ) - \phi _ { n } ( t ) \right) ^ { 2 } d t \right) ^ { 1 / 2 } + \left( \int \left( \phi _ { n } ( t ) - \phi ( t ) \right) ^ { 2 } d t \right) ^ { 1 / 2 }$$
    Since as $n \rightarrow \infty$:
    $$\int \left( f ( t ) - \phi _ { n } ( t ) \right) ^ { 2 } d t \rightarrow 0$$
    Therefore, $\phi_n$ cannot coverge to $\phi$ in $C_2[a,b]$. 
\end{flushleft}
\item An explicit linear functional on $\R^2$ whose null space is spanned by the vector $(1,2)$.
\item An explicit convex functional on $\R^2$.
\item A function in $C_1[0,1]$ but not $C_\infty[0,1]$. 
\item A function in $C_1[0,1]$ but not $C_2[0,1]$.
\item A linear function that is continuous on $C_\infty[0,1]$ but not on $C_1[0,1]$.
\item A continuous operator that is not compact.
\item A compact operator that has no eigenvalues.
\item A continuous operator that has no eigenvalues.
\item An operator from a space to itself that is not self-adjoint.
\item An operator $A:\ell_2\rightarrow \ell_2$ whose eigenvalues include $1$ and $2$.
\item A non-trivial operator $A:\ell_2\rightarrow \ell_2$ whose eigenvalues are all smaller than $1/2$.
\item an inner product space $V$ and an orthonormal system $\{e_i\}$ such that $V$ contains no non-zero element orthogonal to every $e_i$, even though $\{e_i\}$ does not span $V$.
\item A nested sequence of closed non-empty sets $C_i$ in $\R$ such that $\cap C_i=\emptyset$.
\item A nested sequence $\{E_n\}$ of nonempty closed bounded convex sets in a Banach space $V$ (of your choice) such that $\cap_n E_n=\emptyset$.
\item A basis for $C_2[0,\pi]$.
\item An element of $\ell_\infty$ that cannot be written as $\sum_i^\infty a_i e_i$, where $a_i\in \R$ and $e_i$ is the sequence $(0,\ldots, 0, 1, 0, \ldots)$ where the 1 is in the $i^{th}$ place.
\item An continuous operator $A$ that is invertible such that $A^{-1}$ is not continuous.
\item Recall that the \emph{support} of a function is the closure of the set of points where it has non-zero values. Suppose $f\in C_2[0,1]$ is supported on $[0,1/2]$. Give an example of a non-trivial element of $f^\perp$.
\item Recall that $V$ is reflexive if the canonical mapping from $V$ to $(V^*)^*$ is an isomorphism. Give an example of a non-reflexive space.
\end{enumerate}





\end{document}