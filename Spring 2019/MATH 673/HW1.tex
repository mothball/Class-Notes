\documentclass[12pt,letterpaper,reqno]{amsart}
\usepackage{enumerate}
\usepackage[shortlabels]{enumitem}
\usepackage{graphicx}
\usepackage{amssymb}
\usepackage[normalem]{ulem}
\usepackage{titlesec,bbm, hyperref}
\usepackage{spverbatim} 
\usepackage{esvect}
\usepackage{geometry}
\geometry{letterpaper, portrait, margin=1in}

\newcommand{\R}{\mathbb R}

\begin{document}

\thispagestyle{empty}
\begin{center}\large{
    MATH 673\quad
    HW 1\quad
    Sumanth Ravipati\quad
    January 28, 2019}
\end{center}
\vspace{.25in}

\begin{enumerate}
\item[T1.4] Let $p$ be a fixed point of a map $f$. Given some $\epsilon > 0$, find a geometric condition under which all points $x$ in $N(p)$ are in the basin of $p$. Use cobweb plot analysis to explain your reasoning.
\begin{flushleft}
One condition is that $f$ lies strictly between the lines $y = p$ and $y = x$ in $(p - \epsilon,
p + \epsilon)$.
\end{flushleft}

\item [T1.5] The map $f(x) = 2x^2 - 5x$ on $\R$ has fixed points at $x=0$ and $x=3$. Find a period-two orbit for $f$ by solving $f^2(x) = x$ for $x$.
\begin{flushleft}
$ \{1+\sqrt{2}, 1-\sqrt{2}\} $
\end{flushleft}

\item [T1.6] Verify the statements in the previous paragraph by solving the fixed points and period-two points of $g_a(x)$ and evaluating their stability.
\begin{flushleft}

\end{flushleft}

\item [1.2]
\begin{enumerate}
\item[(a)] Let $f(x) = x - x^2$. Show that $x=0$ is a fixed point of $f$, and describe the dynamical behavior of points near $0$.
\begin{flushleft}

\end{flushleft}
\item[(b)] Let $g(x) = \tan x$, $-\pi/2 < x < \pi/2$. Show that $x=0$ is a fixed point of $g$, and describe the dynamical behavior of points near $0$.
\begin{flushleft}

\end{flushleft}
\item[(c)] Give an example of a function $h$ for which $h'(0) = 1$ and $x=0$ is an attracting point.
\begin{flushleft}

\end{flushleft}
\item[(d)] Give an example of a function $h$ for which $h'(0) = 1$ and $x=0$ is a repelling fixed point.
\begin{flushleft}

\end{flushleft}
\end{enumerate}

\item [1.3] Let $f(x) = x^3 + x$. Find all fixed points of $f$ and decide whether they are sinks or sources. You will have to work without Theorem 1.5, which does not apply.
\begin{flushleft}
$x=0$ is a source.
\end{flushleft}

\item [1.5] Is the period-two orbit of the map $f(x) = 2x^2 - 5x$ on $\R$ a sink, a source, or neither? See Exercise T1.5.
\begin{flushleft}
Source.
\end{flushleft}

\end{enumerate}
\end{document}